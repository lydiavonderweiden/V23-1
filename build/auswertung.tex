\section{Auswertung}
\label{sec:Auswertung}

\subsection{Messung der magnetischen Flussdichte bei verschiedenen Abständen}
Das Magnetfeld der verwendeten Spule wurde mit der Hallsonde für verschiedene Abstände vermessen. Die Messwerte der magnetischen Flussdichte $B$ sind in Tabelle \ref{tab:magnetfeld} aufgelistet. Hierbei ist $a$ der Abstand zum Mittelpunkt der Spule in $\si{\milli\metre}$. 
Die zugehörige grafische Darstellung dieser Messungen ist in Abbildung \ref{fig:magnetfeld} dargestellt.

\begin{figure}[H]
    \centering
    \includegraphics[width=\textwidth]{build/magnetfeld.png}
    \caption{Grafische Darstellung der Flussdichte Messung.}
    \label{fig:magnetfeld}
\end{figure}

\begin{table}
    \begin{tabular}{c|c c}
        \toprule
        {} &  Abstand $a \:/\/: \si{\milli\metre}$ &  Feldstärke $B \:/\: \si{\milli\tesla}$\\
        \midrule
        0  &     37 &           0 \\
        1  &     32 &           1 \\
        2  &     27 &           3 \\
        3  &     22 &          11 \\
        4  &     17 &          46 \\
        5  &     14 &          91 \\
        6  &     12 &         174 \\
        7  &      9 &         284 \\
        8  &      7 &         369 \\
        9  &      5 &         410 \\
        10 &      4 &         424 \\
        11 &      3 &         433 \\
        12 &      2 &         439 \\
        13 &      1 &         440 \\
        14 &      1 &         443 \\
        15 &      0 &         444 \\
        16 &      0 &         444 \\
        17 &     -0 &         444 \\
        18 &     -1 &         443 \\
        19 &     -1 &         440 \\
        20 &     -2 &         439 \\
        21 &     -3 &         431 \\
        22 &     -4 &         423 \\
        23 &     -5 &         411 \\
        24 &     -6 &         393 \\
        25 &     -7 &         371 \\
        26 &     -8 &         343 \\
        27 &    -10 &         266 \\
        28 &    -12 &         181 \\
        29 &    -14 &         102 \\
        30 &    -16 &          59 \\
        31 &    -18 &          33 \\
        32 &    -20 &          18 \\
        33 &    -23 &           7 \\
        34 &    -26 &           3 \\
        35 &    -28 &           1 \\
        36 &    -33 &           0 \\
        \bottomrule
    \end{tabular}
    \caption{Grafische Darstellung der Flussdichte Messung.}
    \label{tab:magnetfeld}
\end{table}


\subsection{Bestimmung der Faraday Rotation}

\subsection{Bestimmung der effektiven Masse}
\begin{table}
    \centering
    \begin{tabular}{lrrrr}
        \toprule
        {} &  $\lambda \:/\: \si{\micro\metre}$ &  $\theta_1 \:/\: °$ &  $\theta_2 \:/\: °$ &  $\theta_n  \:/\: °$ \\
        \midrule
        0 &   1,060 &   63,417 &   86,033 &    2,213 \\
        1 &   1,290 &   64,583 &   81,000 &    1,606 \\
        2 &   1,450 &   66,967 &   80,083 &    1,283 \\
        3 &   1,720 &   66,017 &   76,083 &    0,985 \\
        4 &   1,960 &   61,917 &   69,267 &    0,719 \\
        5 &   2,156 &   60,083 &   65,250 &    0,506 \\
        6 &   2,340 &   41,083 &   43,050 &    0,192 \\
        7 &   2,510 &   28,250 &   32,000 &    0,367 \\
        8 &   2,650 &   54,817 &   60,250 &    0,532 \\
        \bottomrule
    \end{tabular}
    \caption{Hochrein}
    \label{tab:hochrein}
\end{table}

\begin{table}
    \centering
    \begin{tabular}{lrrrr}
        \toprule
        {} &  $\lambda \:/\: \si{\micro\metre}$ &  $\theta_1 \:/\: °$ &  $\theta_2 \:/\: °$ &  $\theta_n  \:/\: °$ \\
        \midrule
        0 &   1,060 &   71,233 &   80,017 &    3,389 \\
        1 &   1,290 &   69,933 &   78,783 &    3,414 \\
        2 &   1,450 &   70,483 &   79,033 &    3,299 \\
        3 &   1,720 &   68,317 &   76,733 &    3,247 \\
        4 &   1,960 &   62,033 &   72,883 &    4,186 \\
        5 &   2,156 &   58,067 &   68,983 &    4,212 \\
        6 &   2,340 &   37,767 &   51,017 &    5,112 \\
        7 &   2,510 &   13,633 &   38,750 &    9,690 \\
        8 &   2,650 &   50,483 &   67,067 &    6,398 \\
        \bottomrule
    \end{tabular}
    \caption{N dotiert}
    \label{tab:ndotiert}
\end{table}


\begin{table}
    \centering
    \begin{tabular}{lrrrr}
        \toprule
        {} &  $\lambda \:/\: \si{\micro\metre}$ &  $\theta_1 \:/\: °$ &  $\theta_2 \:/\: °$ &  $\theta_n  \:/\: °$ \\
        \midrule
        0 &   1,060 &   72,050 &   83,050 &    4,044 \\
        1 &   1,290 &   70,683 &   77,383 &    2,463 \\
        2 &   1,450 &   72,017 &   77,983 &    2,194 \\
        3 &   1,720 &   70,583 &   78,033 &    2,739 \\
        4 &   1,960 &   65,583 &   70,050 &    1,642 \\
        5 &   2,156 &   61,017 &   68,467 &    2,739 \\
        6 &   2,340 &   39,533 &   48,033 &    3,125 \\
        7 &   2,510 &   17,967 &   33,767 &    5,809 \\
        8 &   2,650 &   51,567 &   64,767 &    4,853 \\
        \bottomrule
    \end{tabular}
    \caption{leicht N dotiert}
    \label{tab:leichtndotiert}
\end{table}

\begin{table}
    \centering
    \begin{tabular}{lrrrr}
        \toprule
        {} &  $\lambda \:/\: \si{\micro\metre}$ &  $\frac{\theta}{d}_\text{rein}$ &  $\frac{\theta}{d}_\text{dot}$ &  $\frac{\theta}{d}_\text{leichtdot}$ \\
        \midrule
        0 &   1,060 &         0,433 &        2,615 &              2,974 \\
        1 &   1,290 &         0,314 &        2,635 &              1,811 \\
        2 &   1,450 &         0,251 &        2,545 &              1,613 \\
        3 &   1,720 &         0,193 &        2,506 &              2,014 \\
        4 &   1,960 &         0,141 &        3,230 &              1,207 \\
        5 &   2,156 &         0,099 &        3,250 &              2,014 \\
        6 &   2,340 &         0,038 &        3,944 &              2,298 \\
        7 &   2,510 &         0,072 &        7,477 &              4,271 \\
        8 &   2,650 &         0,104 &        4,937 &              3,568 \\
        \bottomrule
    \end{tabular}
    \caption{$\theta/L$}
    \label{tab:thetaL}
\end{table}


\begin{figure}[H]
    \centering
    \includegraphics[width=\textwidth]{build/plot.pdf}
    \caption{Plots. }
    \label{fig:plots}
\end{figure}