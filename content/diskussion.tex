\section{Diskussion}
\label{sec:diskussion}

\subsection{Wasserstoffatom}

Die diskreten Energieniveaus des Wasserstoffatoms werden gut durch den Kugelresonator simuliert, da das Frequenzspektrum ausgepräkte und scharfe Peaks besitzt. 
Die Messungen der Winkelverteilungen für alle betrachteten Peaks sind in diesem Abschnitt auch gut mit den vorhergesagten theoretischen Orbitalen vereinbar. Das $3D$ Orbital des Wasserstoffatoms konnte gut aufgelöst werden.
Aus diesem Grund ist der Kugelresonator ein gutes Modell für die den winkelabhängigen Teil der Wellenfunktion eines Wasserstoffatoms. Die Radialkomponente der Gesamtwellenfunktion kann ein Kugelresonator nicht modellieren.  
Des Weiteren deckt das Auspalten der Peaks im Spektrum durch das Einführen einer Blende innerhalb des Resonators die theoretischen Erwartungen. Eine Blende bricht die Kugelsymmetrie und löst damit die Entartung in $m$ auf. 
\subsection{Winkelverteilung des Wasserstoffmoleküls}

Bei der gemessenen Winkelverteilung der gekoppelten Kugelresonatoren konnte nur das dritte Peak eindeutig zu einem Legendrepolynom zugeordnet werden (siehe Abbildung \ref{fig:h2_2}). 
Dies kann zum Teil damit begründet werden, dass eines der Peaks bzw. beide Peaks sich in Mischzuständen befinden und damit eine Summe aus verschiedenen Legendrepolynomen darstellen. 
Des Weiteren konnte bei diesem größeren und komplexeren Aufbau die Winkel nicht präzise gemessen werden wodurch große Messfehler entstanden sein könnten. \\
Der Zustand des dritten Peaks konnte jedoch eindeutig bestimmt werden. Dies kann damit begründet werden, dass sich die nicht bindenden Zustände besser herauskristallisieren lassen könnten. 
\subsection{Festkörper Modellierung}

Die Messergebnisse zu den Zylinderketten, welche einen eindimensionalen Festkörper simulieren sollen, folgen im Allgemeinen die theoretischen Erwartungen. Die Bandstruktur bei der Resonatorkette ohne ein Defekt weißt genau den erwarteten Verlauf auf. 
Die Anzahl der scharfen Peaks in jedem Band entspricht der Anzahl der Zylinder in der Kette. Das kann gut mit den Elektronenzuständen in einem Festkörper verglichen werden. Hierbei ist der Spin der Elektronen zu vernachlässigen und von einem Elektron pro Zylinder auszugehen. \\
Die Messungen zu der modifizierten Resonatorkette entsprechen ebenfalls den Erwartungen. Durch alternierende Blenden in der Kette kommt es zu Veränderungen innerhalb eines Bandes. Es entstehen hierdurch beispielsweise Bandlücken. 
Durch alternierende Zylinderlängen haben sich die Bänder verschoben. Es kam also zu Veränderungen der gesamten Bandstruktur. Diese Veränderungen entsprechen den tatsächlichen Beobachtungen bei Festkörpern. 