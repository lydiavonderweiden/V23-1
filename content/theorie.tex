\section{Zielsetzung}
\label{sec:ziel}

Ziel des Versuchs ist es quantenmechanische Strukturen wie das Wasserstoffatom, Wasserstoffmolekül und die Bandstruktur in Festkörpern mit Hilfe von Analogien in der Akustik zu untersuchen und die Gemeinsamkeiten und Grenzen dieser Analogien zu untersuchen. 

\section{Theorie}
\label{sec:theorie}

Für die quantenmechanischen Modellen können Analogien mit Hilfe der Akustik geschaffen werden, im Folgenden werden die quantenmechanischen Grundlagen für die einzelnen Modelle erläutert und die Gemeinsamkeiten und Unterschiede zu den akustischen Experimenten benannt und begründet. 

\subsection{Das Wasserstoffatom}
\label{sec:H}

Das Wasserstoffatom ist das simpelste Atom. Es besteht aus einem Proton im Kern und einem Elektron in der Hülle. Die zeitunabhängige Schrödingergleichung lautet:

\begin{equation}
    E \, \Psi \! \left( \vec{r} \right) = \hat{H} \, \Psi \! \left( \vec{r} \right)
    \label{eqn:schroedinger}
\end{equation}

Dabei ist $\Psi \! \left( \vec{r} \right)$ ist die Wellenfunktion, $E$ die Gesamtenergie und $\hat{H}$ der Hamiltonoperator. Für ein Elektron im Wasserstoffatom lautet $\hat{H}$:

\begin{equation}
    \hat{H} = - \frac{\hbar^2}{2 m} \Laplace - \frac{e^2}{4 \pi \epsilon_0 r}
    \label{eqn:H_H}
\end{equation}

Hierbei ist $\hbar$ das gekürzte planksche Wirkungsquantum, $m$ die Masse des Elektrons, $e$ die Elementarladung und $\epsilon_0$ die elektrische Feldkonstante. In Kugelkoordinaten lautet der Laplace-Operator $\Laplace$ für eine Beispielfunktion $f$:

\begin{equation}
    \Laplace f = \frac{1}{r^2} \frac{\partial}{\partial r} \left( r^2 \frac{\partial f}{\partial r} \right) + \frac{1}{r^2 \sin \theta} \frac{\partial}{\partial \theta} \left( \sin \theta \frac{\partial f}{\partial \theta} \right) + \frac{1}{r^2 \sin^2 \theta} \frac{\partial^2 f}{\partial \phi^2}
    \label{eqn:laplace}
\end{equation}

Um die Schrödingergleichung zu lösen wird die Wellenfunktion $\Psi$ mit dem Seperationsansatz in einen Radialteil $R_{nl}$ und einen Winkelanteil $\Phi_{lm}$ aufgeteilt:

\begin{equation}
    \Psi_{nlm} (\vec{r}) = R_{nl}(r) \, \Phi_{lm}(\theta, \phi)
    \label{eqn:seperation}
\end{equation}

Für das Wasserstoffatom gibt es 3 Quantenzahlen namens $n, l, m$. Dabei ist $n$ die Hauptquantenzahl, $l$ die Nebenquantenzahl und $m$ die Magnetquantenzahl. Für die Quantenzahlen gilt

\begin{align*}
    n &\in \mathbb{N} \\
    l &\in \mathbb{N}_0 \\
    m &\in \mathbb{Z}
\end{align*}

und

\begin{align*}
    l &< n \\
    |m| &\leq l .
\end{align*}

Mit dem Seperationsansatz entstehen zwei entkoppelte Differentialgleichungen.

\subsection{Das Wasserstoffmolekül}
\label{sec:H2}



\subsection{Der 1-dim Festkörper}
\label{sec:festkoerper}



\subsection{Grundlagen der Akustik}
\label{sec:akustik}



\subsection{Analogie zum Wasserstoffatom und -molekül}
\label{sec:analogien}

Im Folgenden werden die Analogien zwischen akustischen Experimenten und dem Modell des Wasserstoffatom und Wasserstoffmolekül dargestellt.

\subsubsection{Wasserstoffatom}
\label{sec:ana-H}



\subsubsection{Wasserstoffmolekül}
\label{sec:ana-H2}



\subsection{Analogie zum 1-dim Festkörpers}
\label{ana-fest}

