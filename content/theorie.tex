\section{Zielsetzung}
\label{sec:ziel}

Ziel des Versuchs ist es quantenmechanische Strukturen wie das Wasserstoffatom, Wasserstoffmolekül und die Bandstruktur in Festkörpern mit Hilfe von Analogien in der Akustik zu untersuchen und die Gemeinsamkeiten und Grenzen dieser Analogien zu untersuchen. 

\section{Theorie}
\label{sec:theorie}

Für die quantenmechanischen Modellen können Analogien mit Hilfe der Akustik geschaffen werden, im Folgenden werden die quantenmechanischen Grundlagen für die einzelnen Modelle erläutert und die Gemeinsamkeiten und Unterschiede zu den akustischen Experimenten benannt und begründet. 

\subsection{Das Wasserstoffatom}
\label{sec:H}



\subsection{Das Wasserstoffmolekül}
\label{sec:H2}



\subsection{Der 1-dim Festkörper}
\label{sec:festkoerper}



\subsection{Grundlagen der Akustik}
\label{sec:akustik}



\subsection{Analogie zum Wasserstoffatom und -molekül}
\label{sec:analogien}

Im Folgenden werden die Analogien zwischen akustischen Experimenten und dem Modell des Wasserstoffatom und Wasserstoffmolekül dargestellt.

\subsubsection{Wasserstoffatom}
\label{sec:ana-H}



\subsubsection{Wasserstoffmolekül}
\label{sec:ana-H2}



\subsection{Analogie zum 1-dim Festkörpers}
\label{ana-fest}

